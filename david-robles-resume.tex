\documentclass[10pt]{article}

\usepackage{calc}
\reversemarginpar

%
%         PAPER SIZE, PAGE NUMBER, AND DOCUMENT LAYOUT NOTES:
%
% The next \usepackage line changes the layout for CV style section
% headings as marginal notes. It also sets up the paper size as either
% letter or A4. By default, letter was used. If A4 paper is desired,
% comment out the letterpaper lines and uncomment the a4paper lines.
%
% As you can see, the margin widths and section title widths can be
% easily adjusted.
%
% ALSO: Notice that the includefoot option can be commented OUT in order
% to put the PAGE NUMBER *IN* the bottom margin. This will make the
% effective text area larger.
%
% IF YOU WISH TO REMOVE THE ``of LASTPAGE'' next to each page number,
% see the note about the +LP and -LP lines below. Comment out the +LP
% and uncomment the -LP.
%
% IF YOU WISH TO REMOVE PAGE NUMBERS, be sure that the includefoot line
% is uncommented and ALSO uncomment the \pagestyle{empty} a few lines
% below.
%

%% Use these lines for letter-sized paper
\usepackage[paper=letterpaper,
            %includefoot, % Uncomment to put page number above margin
            marginparwidth=1.2in,     % Length of section titles
            marginparsep=.05in,       % Space between titles and text
            margin=1in,               % 1 inch margins
            includemp]{geometry}

%% Use these lines for A4-sized paper
%\usepackage[paper=a4paper,
%            %includefoot, % Uncomment to put page number above margin
%            marginparwidth=30.5mm,    % Length of section titles
%            marginparsep=1.5mm,       % Space between titles and text
%            margin=25mm,              % 25mm margins
%            includemp]{geometry}

%% More layout: Get rid of indenting throughout entire document
\setlength{\parindent}{0in}

%% This gives us fun enumeration environments. compactitem will be nice.
\usepackage{paralist}

%% Reference the last page in the page number
%
% NOTE: comment the +LP line and uncomment the -LP line to have page
%       numbers without the ``of ##'' last page reference)
%
% NOTE: uncomment the \pagestyle{empty} line to get rid of all page
%       numbers (make sure includefoot is commented out above)
%
\usepackage{fancyhdr,lastpage}
\pagestyle{fancy}
%\pagestyle{empty}      % Uncomment this to get rid of page numbers
\fancyhf{}\renewcommand{\headrulewidth}{0pt}
\fancyfootoffset{\marginparsep+\marginparwidth}
\newlength{\footpageshift}
\setlength{\footpageshift}
          {0.5\textwidth+0.5\marginparsep+0.5\marginparwidth-2in}
\lfoot{\hspace{\footpageshift}%
       \parbox{4in}{\, \hfill %
                    \arabic{page} of \protect\pageref*{LastPage} % +LP
%                    \arabic{page}                               % -LP
                    \hfill \,}}

% Finally, give us PDF bookmarks
\usepackage{color,hyperref}
\definecolor{darkblue}{rgb}{0.0,0.0,0.3}
\hypersetup{colorlinks,breaklinks,
            linkcolor=darkblue,urlcolor=darkblue,
            anchorcolor=darkblue,citecolor=darkblue}

%%%%%%%%%%%%%%%%%%%%%%%% End Document Setup %%%%%%%%%%%%%%%%%%%%%%%%%%%%


%%%%%%%%%%%%%%%%%%%%%%%%%%% Helper Commands %%%%%%%%%%%%%%%%%%%%%%%%%%%%

% The title (name) with a horizontal rule under it
%
% Usage: \makeheading{name}
%
% Place at top of document. It should be the first thing.
\newcommand{\makeheading}[1]%
        {\hspace*{-\marginparsep minus \marginparwidth}%
         \begin{minipage}[t]{\textwidth+\marginparwidth+\marginparsep}%
                {\large \bfseries #1}\\[-0.15\baselineskip]%
                 \rule{\columnwidth}{1pt}%
         \end{minipage}}

% The section headings
%
% Usage: \section{section name}
%
% Follow this section IMMEDIATELY with the first line of the section
% text. Do not put whitespace in between. That is, do this:
%
%       \section{My Information}
%       Here is my information.
%
% and NOT this:
%
%       \section{My Information}
%
%       Here is my information.
%
% Otherwise the top of the section header will not line up with the top
% of the section. Of course, using a single comment character (%) on
% empty lines allows for the function of the first example with the
% readability of the second example.
\renewcommand{\section}[2]%
        {\pagebreak[2]\vspace{1.3\baselineskip}%
         \phantomsection\addcontentsline{toc}{section}{#1}%
         \hspace{0in}%
         \marginpar{
         \raggedright \scshape #1}#2}

% An itemize-style list with lots of space between items
\newenvironment{outerlist}[1][\enskip\textbullet]%
        {\begin{itemize}[#1]}{\end{itemize}%
         \vspace{-.6\baselineskip}}

% An environment IDENTICAL to outerlist that has better pre-list spacing
% when used as the first thing in a \section 
\newenvironment{lonelist}[1][\enskip\textbullet]%
        {\vspace{-\baselineskip}\begin{list}{#1}{%
        \setlength{\partopsep}{0pt}%
        \setlength{\topsep}{0pt}}}
        {\end{list}\vspace{-.6\baselineskip}}

% An itemize-style list with little space between items
\newenvironment{innerlist}[1][\enskip\textbullet]%
        {\begin{compactitem}[#1]}{\end{compactitem}}

% To add some paragraph space between lines.
% This also tells LaTeX to preferably break a page on one of these gaps
% if there is a needed pagebreak nearby.
\newcommand{\blankline}{\quad\pagebreak[2]}

%%%%%%%%%%%%%%%%%%%%%%%% End Helper Commands %%%%%%%%%%%%%%%%%%%%%%%%%%%

%%%%%%%%%%%%%%%%%%%%%%%%% Begin CV Document %%%%%%%%%%%%%%%%%%%%%%%%%%%%

\begin{document}
\makeheading{David Robles}

%%%%%%%%%%%%%%%%%
\section{Contact}
%%%%%%%%%%%%%%%%%

\newlength{\rcollength}\setlength{\rcollength}{2.8in}%
%
\begin{tabular}{ l l }
Phone:    & (415) 515-4764 \\
Email:    & \href{mailto:drobles@gmail.com}{drobles@gmail.com} \\
Twitter:  & @davidrobles \\
Linkedin: & \href{http://www.linkedin.com/in/drobles}{http://www.linkedin.com/in/drobles} \\
\end{tabular}

%%%%%%%%%%%%%%%%%
\section{Summary}
%%%%%%%%%%%%%%%%%
%
Software Engineer with strong focus on tackling engineering challenges, writing quality software and
with strong grasp of software architecture, advanced software engineering principles and design
patterns. \\

I recently completed an MSc and PhD in Computer Science from the University of \mbox{Essex}, where I
conducted research focused on Reinforcement Learning, Simulation-Based Search, Evolutionary
Algorithms and Games. During this time I was also a Teaching Assistant in several courses in
programming and web application development. \\

I'm currently a Software Engineer at dotloop, where I work primarily as a Java backend developer
responsible for designing and implementing new features and RESTful APIs. In my free time, I love
programming in Ruby, Rails, Python, and more recently I've been diving into functional
programming.

%%%%%%%%%%%%%%%%%%%
\section{Education}
%%%%%%%%%%%%%%%%%%%
%
\textbf{PhD Computer Science} \\
\href{http://www.essex.ac.uk/}{University of Essex} \\
Colchester, United Kingdom \\
2009 - 2013 \\
\\
Thesis: ``Simulation-Based Search and Learning in Games'' \\

\textbf{MSc Computer Science} \\
\href{http://www.essex.ac.uk/}{University of Essex} \\
Colchester, United Kingdom \\
2007 - 2008 \\
\\
Software engineering theme, passed with distinction. \\

\textbf{BSc Information Systems Engineering} \\
\href{http://www.itesm.edu/}{Institute of Technology and Higher Education of Monterrey} \\
Hermosillo, Mexico \\
2001 - 2006

%%%%%%%%%%%%%%%%%%%%%%%%%
\section{Experience}
%%%%%%%%%%%%%%%%%%%%%%%%%
%
\textbf{Software Engineer} \\
DotLoop \\
San Francisco, CA \\
January 2013 - Present \\
\\
- Java backend developer. \\
- Design and implementation of new features and RESTful APIs in Spring MVC. \\
\\
\newpage
%%%%%%%%%%%%%%%%%%%%%%%%%%%%%%%%%%%%%%%%%%%%%%%%%%%%%%%%%%%%%%%%%%%%%%%%%%%%%%%%%%%%%%%%%%%%%%%%%
\textbf{Teaching Assistant} \\
University of Essex \\
Colchester, United Kingdom \\
October 2009 - March 2012\\
\\
Preparation, marking and teaching of classes and laboratories for the following undergraduate and
postgraduate courses:
\\ \\
CE705 - Programming in Java, Autumn 2011. \\
CE320 - Large Scale Software Scale Systems and Extreme Programming, Autumn 2011. \\
CE218 - Computer Game Programming, Spring 2011. \\
CE152 - Object-Oriented Programming, Spring 2011. \\
CE705 - Programming in Java, Autumn 2010. \\
CE212 - Web Application Programming, Spring 2010. \\
CE112 - Procedural and Object-Oriented Programming, Spring 2010. \\
CE705 - Programming in Java, Autumn 2009. \\
\\
%%%%%%%%%%%%%%%%%%%%%%%%%%%%%%%%%%%%%%%%%%%%%%%%%%%%%%%%%%%%%%%%%%%%%%%%%%%%%%%%%%%%%%%%%%%%%%%%%
\textbf{Systems Administrator} \\
Camisa Development Group \\
Hermosillo, Mexico \\
March 2006 - May 2007  \\
\\
Network Administrator, Windows Server 2003, Active Directory, VPN, CRM Software, Sage SalesLogix.
\\
%%%%%%%%%%%%%%%%
\section{Skills} 
%%%%%%%%%%%%%%%%
%
\\
\textbf{Programming languages}: Java, Ruby, Python, PHP, C. \\
\textbf{Web Development}: Ruby on Rails, Spring MVC, JSP/Servlets, JavaScript, CSS. \\
\textbf{Testing}: JUnit, RSpec, Mockito. \\
\textbf{Databases}: MySQL, PostgreSQL. \\
\textbf{Cloud}: Amazon EC2/S3, Heroku. \\
\textbf{Version Control}: Git, SVN. \\
\textbf{OS}: Unix/Linux, Mac OS X.

%%%%%%%%%%%%%%%%
\section{Awards} 
%%%%%%%%%%%%%%%%
%
Scholarship for MSc and PhD Computer Science degrees from the National Council of Science and
Technology of Mexico.

%%%%%%%%%%%%%%%%%%%%%%%%
\section{Certifications}
%%%%%%%%%%%%%%%%%%%%%%%%
%
Cisco Certified Network Associate (CCNA), February 2006. \\
Microsoft Certified Systems Administrator (MCSA), May 2007. \\

\end{document}

